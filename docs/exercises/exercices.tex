\documentclass[11pt, oneside]{article}   	% use "amsart" instead of "article" for AMSLaTeX format
\usepackage{geometry}                		% See geometry.pdf to learn the layout options. There are lots.
\geometry{letterpaper}                   		% ... or a4paper or a5paper or ... 
%\geometry{landscape}                		% Activate for for rotated page geometry
%\usepackage[parfill]{parskip}    		% Activate to begin paragraphs with an empty line rather than an indent
\usepackage{graphicx}				% Use pdf, png, jpg, or eps§ with pdflatex; use eps in DVI mode
								% TeX will automatically convert eps --> pdf in pdflatex		
\usepackage{amssymb}
\newtheorem{axiom}{Axiom}
\newtheorem{lemma}{Lemma}
\newtheorem{definition}{Definition}

\title{Brief Article}
\author{The Author}
%\date{}							% Activate to display a given date or no date

\begin{document}
\maketitle
%\section{}
%\subsection{}
We will have a set of points and a set of lines. We also have a relation $p \in L$, some points belong to some lines
\begin{definition}
Two lines are parallel if they have no points in common.
\end{definition}
\begin{axiom}
Any two points determine a line. that is if $p1\ne p2$ then there exists a unique line $L$ so that $p_{i} \in L$.
\end{axiom}

\begin{axiom}
A line has exactly two points.
\end{axiom}

\begin{axiom}
for any point $p$ and line $L$ with $p \not \in L$  there is exactly one line  through $p$ that is parallel to $L$.
\end{axiom}

\begin{axiom}
There are at least three points
\end{axiom}


\begin{lemma}
There exist at least 4 points.
\end{lemma}
\begin{lemma}
ether exist at least six lines.
\end{lemma}
\begin{lemma}
There exist at most 4 points.
\end{lemma}

\end{document}  