\section{Predicate calculus}\label{sec:predicatecalculus}

Nice as it might be, propositional calculus is not complete enough to express what we want. Here are some example of statements that we would like to deal with
\begin{itemize}
\item The equation $x^2+x+1=0$ does not have any solution.
\item Some people like bread and some do not.
\item If $a,b, c$ are natural numbers,$a|b \land a | c \rightarrow a | (b+c)$.
\item Any differentiable function is continuous.
\end{itemize}

All these require more general notion than that of a proposition, that of a predicate. 
For example $x>0$ might or might not be true depending on $x$. We can view this as a function from $\mathbb{R}$ to the sett of propositions or as a set of propositions, parametrised by $\mathbb{R}$. 

This exactly the meaning of a predicate, it is a collection of propositions parametrised by a context (type). Here are some propositions.

\begin{itemize}
\item P(x): $x^{2}+x+1 =0$ (here x is a real number).
\item P(p): p is a prime. (here p is a natural number)
\item P(x) : x is a man. (here x is an animal)
\item P(x, y) : x >y. (here both x an y are real numbers and so P : $mathbb{R}^{2}\rightarrow Prop.$

\end{itemize}	

Of course you cannot really prove predicates, just statements. Predicates have ``free'' variables and those need to be ``quantified''. There are two quantifiers that bind variables. as with connectors for propositions they have introduction and elimination rules.

\paragraph{\bf Existential Quantifier $\exists$}

The meaning of this quantifier is self explanatory. If $P:U \rightarrow Prop$ is a predicate then $\exists x:U, P(x)$ is a proposition which is true if you can find an x so that $P(x)$ is true. Note that in SPatchCoq you can enter this either by clicking on the symbol or by typing exists.


For example $\exists x:\mathbb{R}, x^{2}+x+1 =0.$ means that the equation $x^{2}+x+1=0$ has a solution. Therefore our first example of the section ``The equation $x^2+x+1=0$ does not have any solution.'' can be written as $\neg (\exists x:\mathbb{R}, x^{2}+x+1 =0).$ 

If we consider the predicate ``P(x) : x likes bread'' on the set People of all people then ``Some people like bread and some do not.'' can be written as $(\exists x:People, P(x))\land (\exists x:People, \neg P(x))$.