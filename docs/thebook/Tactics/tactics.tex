
     
     
     
     
\chapter{Tactics}\label{ch:tactics}

\paragraph{\bf This is trivial.}

This will only work on very easy statements. If it works it will solve the current goal. Try to avoid overuse. Do better than your lecturers.


\paragraph{\bf I cannot prove this.}

If you are stuck this tactic will ``prove'' the current goal. If you use this in a proof at the end if the proof when you try to use Qed you will get the following error
\mess{``Error:Attempt to save a proof with given up goals. If this is really what
you want to do, use Admitted in place of Qed.}
To avoid the error just type Admitted instead of Qed.


\paragraph{\bf Prove left hand side.}
Suppose you want to prove the following goal:
\coq{\cdots
}{
P\lor Q}

The above mentioned tactic will the produce a the following goal

\coq{ }{
P}
 and so you will have to now prove a simple goal.

\paragraph{\bf Prove right hand side.}
Symmetric with the above, suppose you want to prove the following goal:
\coq{ \cdots }{
P\lor Q}

The above mentioned tactic will the produce a the following goal

\coq{ \cdots }{
Q}



\paragraph{\bf Prove VAR in the disjunction.}
This tactic combines the above two. More precisely,
suppose you want to prove the following goal:
\coq{\cdots}{
P\lor Q}

Then applying 
\inp{Prove P in the disjunction.}

will produce the goal 
\coq{\cdots}{P}
while applying
\inp{Prove Q in the disjunction.}

will produce the goal 
\coq{\cdots}{
Q}

\paragraph{\bf Eliminate the conjuction in hypothesis VAR.}

Suppose your goal looks like
\coq{\cdots
Hyp: P\land Q\\
\cdots }{
\cdots}

Then applying 
\inp{Eliminate the conjunction in hypothesis Hyp.}
will produce a goal similar to the one below:

\coq{\cdots\\ 
Hyp: P\\
Hyp0: Q\\
\cdots}{\cdots}
allowing you to use the parts of Hyp independently.


\paragraph{\bf Consider cases based on disjunction in hypothesis VAR.}
Suppose your goal looks like
\coq{\cdots\\ 
Hyp: P\lor Q\\
\cdots \\
}{
\cdots}
Then applying 
\inp{Consider cases based on disjunction in hypothesis Hyp.}
will produce two separate goals similar to the one below:
\coq{\cdots\\ 
Hyp: P\\
\cdots \\
}{
\cdots}

\coq{\cdots\\ 
Hyp: Q\\
\cdots \\
}{
\cdots}

obtaining a proof by cases.

\paragraph{\bf Prove the conjunction in the goal by first proving VAR then VAR.}
Suppose your goal looks like
\coq{\cdots\\ 
}{
P\land Q}
 Then 
 \inp{Prove the conjunction in the goal by first proving P then Q.}
 
 will separate the proof in two different goals
 
 \coq{\cdots\\ 
}{
P}
\coq{\cdots\\ 
}{
Q}

 
\paragraph{\bf Assume VAR then prove VAR.}
Suppose your goal looks like
\coq{\cdots\\ 
}{
P\rightarrow Q}

then 
\inp{Assume P then prove Q.}
will modify the goal to

\coq{\cdots\\ 
P
}{
Q}



\paragraph{\bf Prove both directions of VAR iff VAR.}

Suppose your goal looks like
\coq{\cdots
}{
P\leftrightarrow Q}

then 
\inp{Prove both directions of P iff Q.}
will split the goal into two different goals

\coq{\cdots\\ 
}{
P\rightarrow Q}

\coq{\cdots\\ 
}{
Q\rightarrow P}



\paragraph{\bf Fix an arbitrary element VAR.}

Suppose your goal looks like
\coq{\cdots\\ 
}{ 
\forall x:S, P(x) }

then 
\inp{Fix an arbitrary element a.}
will modify the goal to

\coq{\cdots\\ 
a:S\\
}{
P(a)}

\paragraph{\bf Fix VAR the existentially quantified variable in VAR.}
Suppose your goal looks like
\coq{\cdots\\ 
Hyp:\exists x:S, P(x)\\
}{
\cdots }

then 
\inp{Fix a the existentially quantified variable in Hyp.}
will modify the goal to

\coq{\cdots\\ 
a:S\\
Hyp:P(a)\\
}{
\cdots}


\paragraph{\bf Obtain VAR using variable VAR in the universally quantified hypothesis VAR.}
Suppose your goal looks like
\coq{\cdots\\ 
Hyp:\forall x:S, P(x)\\
}{ 
\cdots }

then 

\inp{Obtain Q using variable a in the universally quantified hypothesis Hyp.}

will attempt to apply the result $P(a)$ to prove the result $Q$.

\paragraph{\bf Prove the existential claim is true for VAR.}

Suppose your goal looks like
\coq{\cdots\\ 
}{
\exists x:S, P(x) }

then 
\inp{Prove the existential claim is true for a.}
will modify the goal to

\coq{\cdots\\ 
}{ 
P(a)}


\inp{Rewrite the goal using VAR.}

Suppose your goal looks like
\coq{\cdots\\ 
Hyp: x= f\\
}{
 P(x) }

then 
\inp{Rewrite the goal using Hyp.}
will replace every occurrence of x in P by f. Similarly if the goal is 

\coq{\cdots\\ 
Hyp: x= f\\
}{ 
 P(f) }

\inp{Rewrite the goal using Hyp.}
will replace every occurrence of f in P by x.  

Finally if Thm is a theorem whose conclusion includes and equality $x=f$ and if the goal of your theorem looks like

\coq{\cdots\\ 
}{
 P(x) }
 Then 
\inp{Rewrite the goal using Thm.}
will replace every occurrence of x in P by f.  



\paragraph{\bf True by arithmetic properties.}

This tactic will attempt to prove the statement by using the ring properties (commutativity, associativity and distributivity) of the natural, integers or reals. 

\paragraph{\bf Claim VAR by rewriting VAR using VAR.}

This is very similar to \inp{Rewrite the goal using VAR.}

The idea is that 
\inp{Claim Q by rewriting Hyp using Thm.}

Will attempt to prove the statement $Q$ by applying the rewritten version of $Hyp$. The rules for $Thm$ are as above.

\paragraph{\bf Claim VAR.}

This is forward proof tactic. 
\inp{Claim P.}

will introduce a new claim, splitting the goal

\coq{\cdots\\ 
}{
Q }

into 

\coq{\cdots\\ 
}{
 P }
and 

\coq{\cdots\\ 
Hyp:P\\
}{
 Q }





\paragraph{\bf Rewrite hypothesis VAR using the definition of VAR.}
If the hypothesis Hyp will involve a previous definition d, then
\inp{Rewrite hypothesis Hyp using the definition of d.}
 will unfold a definition of d inside Hyp.

\paragraph{\bf Apply induction on VAR.}

This is a rather general tactic. It will generally act as an induction omnibus. More precisely

\inp{Apply induction on n.}
will depend on the (inductive) type of n. For example if $n$ is a natural number and the goal is

\coq{\cdots\\ 
}{
 P(n) }
 then 
\inp{Apply induction on n.}

will split the proof into two goals
\coq{\cdots\\ 
}{
 P(0) }
 
 and 
 
 \coq{\cdots\\ 
 IHn:P(n)\\
}{
 P( S n) }
On the other hand if $n$ is an integer, the goal
\coq{\cdots\\ 
}{ 
 P(n) }
will be split into 3 cases

\coq{\cdots\\ 
}{ 
 P(0) }
 
 \coq{\cdots\\ 
 n:positive
}{
 P(Z.pos\  n) }
 
  \coq{\cdots\\ 
 n:negative
}{ 
 P(Z.neg\  n) }

\paragraph{\bf Rewrite goal using the definition of VAR.}
If the goal will involve a previous definition d, then
\inp{Rewrite goal using the definition of d.}
 will unfold a definition of d inside the conclusion of the goal.


\paragraph{\bf obtain VAR applying VAR to VAR}.

\paragraph{\bf Prove by contradiction.}

Assume the goal is:

\coq{\cdots\\ 
}{
 P }
 
 then  
 \inp{Prove by contradiction.} will transform the goal to
 
 \coq{\cdots\\ 
 \neg P\\
}{
 False}


\paragraph{\bf This follows from reflexivity.}
Assume the goal is 
\coq{\cdots\\ 
}{
 a=a }
 then \inp{This follows from reflexivity.} will finish the proof.
 
This follows from symmetry.


\paragraph{\bf Apply result VAR.}
Assume that the goal is
\coq{\cdots\\ Hyp: P->Q \\ ..
}{
 Q }
 
Then \inp{Apply result Hyp} will transform the goal to

 \coq{\cdots\\ Hyp: P->Q \\ ..
}{
 P }

Similarly if there is a theorem whose name is thm and whose conclusion is $ P\rightarrow Q $then
Then \inp{Apply result thm} will transform the goal to

 \coq{\cdots\\ Hyp: P->Q \\ ..
}{
 P }

\paragraph{\bf This follows from assumptions.}
if the goal is 

\coq{\cdots \\ Hyp:P\\ \cdots}{P}

Then the tactic

{\bf This follows from assumptions.}

finishes the proof.


\paragraph{\bf Denote VAR by VAR}
This is a techincal tactic. 

{\bf Denote $expr_0$ by $expr_1$.}

 will modify the goal by adding a hypothesis 
 
 $$H: expr_1 = expr_0.$$
 
 You usually use this to rewrite terms to simply some notations\footnote{We owe the idea of this tactic to Curt Bennett}.
.