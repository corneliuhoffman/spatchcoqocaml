\documentclass[11pt, oneside]{beamer}   	% use "amsart" instead of "article" for AMSLaTeX format
\usepackage{geometry}                		% See geometry.pdf to learn the layout options. There are lots.
\geometry{letterpaper}                   		% ... or a4paper or a5paper or ... 
%\geometry{landscape}                		% Activate for rotated page geometry
%\usepackage[parfill]{parskip}    		% Activate to begin paragraphs with an empty line rather than an indent
\usepackage{graphicx}				% Use pdf, png, jpg, or eps§ with pdflatex; use eps in DVI mode
								% TeX will automatically convert eps --> pdf in pdflatex		
\usepackage{amssymb}

%SetFonts

%SetFonts


\title{Brief Article}
\author{The Author}
%\date{}							% Activate to display a given date or no date

\begin{document}

%\section{}
%\subsection{}
\frame{
Individual Reflective Commentary: Assignment Brief
Overview

In this final assignment for Mathematical Modelling and Problem Solving you are asked to produce a 500 ? 1,250 word narrative that reflects upon your experience of undertaking this module. It is essential this narrative is reflective and based upon your personal experiences; in particular, based upon these experiences you may find it helpful to try and address the following points:
\begin{itemize}
\item What you found challenging and why.
\item What worked well and what didn?t; what will you (or future teams you are in) now
do differently as a result?
\item How you have developed your skills, outlook and ideas through the activities
undertaken in this module.
\item How you have worked effectively or ineffectively: either individually or how you
contributed to team based tasks.
\item Skills areas that you need/wish to develop in the future (either as part of your
University programme or outside of it), why, and how you intend to go about this.
\item How the module has contributed your future careers plans.
Your reflective commentary is personal but it is not an evaluation of the module, it is an analysis of your own personal experiences.
\end{itemize}}
\frame{
Key Information 

1. Your Reflective Commentary must be uploaded to Canvas by 1700 on the 10 January 2018. This is after the Christmas break to allow you time to reflect upon your experiences.

2. This assignment contributes 20\% of your overall module grade. This is a personal piece of writing, and so grading will reflect that, however, indicative grading criteriafor this assignment are available for download from Canvas.

3. The word limit (500-1,250 words) is indicative. You should not worry if your
submission falls slightly outside this range as it will not be closely enforced, however please remember an important skill is the ability to write concisely.}

\frame{
Advice and Guidance
As reflective writing is a personal account of an individual?s experiences, beliefs and attitudes, no previous examples are available from 1MMPS. Instead, guidance has been provided on writing reflectively along with some general examples for you to review ? these are available on Canvas. The presentation used to discuss reflective writing in Week 11 will be available on Canvas for download.
Assessment
While your reflective commentary will contribute 20\% of your module grade, as noted in the Module Description a number of other factors will be taken into account when determining your mark for this component. These include:
\begin{itemize}
\item Your four individual diaries for each projects.
\item Your contribution to each project (as assessed by your peers in their diaries).
\item Your contribution to 1MMPS as a whole (as assessed by your peers via the survey
and via the PGTAs).
\end{itemize}
Therefore, you need to be aware that the best piece of reflective writing may not necessarily receive the highest grading.
Given that your mark for this assignment will be drawn from a number of different aspects, you will only receive brief feedback on the reflective writing aspect of your submission. Detailed generic feedback will be available.
}


\end{document}  