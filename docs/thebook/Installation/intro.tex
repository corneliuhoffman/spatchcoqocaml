\preface
\epigraph{Proofs are to mathematics what spelling (or even calligraphy) is to poetry. Mathematical works do consist of proofs, just as poems do consist of characters.}{Vladimir Arnold}




So reader, how would you describe Mathematics?

 Ask  the layperson and you will get an image of a series of dreadful  formulas and calculations,  perhaps useful but  so complicated  that they fit the old Arthur C Clarke saying ``Any sufficiently advanced technology is indistinguishable from magic.''

Ask  her the same question and a mathematician will talk of abstract reasoning,  beauty and elegance. Indeed  a hundred years ago Henry Poincare was musing ``The useful combinations are precisely the most beautiful, I mean those best able to charm this special sensibility that all mathematicians know, but of which the profane are so ignorant as often to be tempted to smile at it.''

Most mathematicians will  say that what is important in Mathematics Education is not necessary learning  long calculations but understanding how to reason correctly and using abstraction in  problem solving. I think Mathematics Education should reflect both notions. While one should not discard the  value of  computational tools, we should also teach the beauty of  proofs and reasoning. The approach is as old as teaching  Mathematics. Euclid's Elements, the oldest textbook in the western world is a collection of proofs and constructions. Euclid's proofs  have lead the teaching of mathematics for much of last two thousand years, it stands second only to the Bible in having been printed in more than one thousand editions, 

However,  teaching  proofs is loosing ground  in the modern Mathematics curriculum. For example,  in the UK   Advance level exam from 1957,  7 of the 10 questions involved a small proof. In the 2016 equivalent (C4 AQA test), only 16 out of 75 points included proofs. The result is that many students start university viewing mathematics  as a series of cookbook methods and computations. One of the most serious stumbling blocks in University Mathematics remains the lack of exposure to proofs. One can find many explanations for this but one of the most important is cost. 

This text is an attempt to address this issue. Of course countless textbooks make the same claim  so writing yet another standard one would be  pointless. I will take a rather  non-standard approach.

In the last few decades computer aided education took flight. Computer Algebra systems such as Maple, Mathematica, Mathlab, Maxima, Sage and so on have permeated the curriculum providing examples, modelling and  automated assessment tools. The teaching of proofs and abstractions have seen almost none of these.

This is surprising since computer assisted proofs are almost as old as computers. Already in 1954  Martin Davis encoded Presburger's arithmetic and managed to prove that the sum of even numbers is even. More importantly, a few years later  Newell, Simon and  Shaw wrote  the ``Logic Theorist'', a first order logic solver that managed to prove 38 of the theorems in Russell and Whitehead's  ``Principia Mathematica''. PROLOG in the 70's offered a reasonably simple context to verify first order logic.

Until quite recently though, proof assistants belonged to the world of Computer Science. Very little of the discoveries in the domain crossed  into mathematics or mathematics education.

In the 80's a plethora of Proof assistants appeared. Initially, they mimicked the standard language of mathematics (see for example Mizar) but soon they simplified notation for the sake of efficiency. Despite attempts by some developers (such as the decorative mode Isar for Isabelle), most proof assistants remain beyond the reach of a beginning mathematics student.

I have made several attempts to teach Mathematics with the help of Isabelle and Coq and they all had  modest success. The syntax proved to be too much for the students. The solution we found was to develop a separate interface for Coq that will separate the student from both the terseness and the automation power of the theorem prover and will provide an accessible and interactive syntax. The resulting product is called Spatchcoq, after the method of ``butterflying '' a chicken prior to cooking. I hope  the wonderful authors of Coq will forgive the little inside joke.

The idea of the book is to teach discrete mathematics (the standard way of introducing proofs) with the help of Spatchcoq. I encourage the reader  to download the software using the instructions in Appendix~\ref{ch:thesoftware}. 


I will slowly introduce the software using basic proofs methods in Chapter~\ref{chap:Basicproof}  and give many examples. The inpatient reader can skip to Appendix~\ref{ch:tactics} to get short descriptions of the tactics, respectively to Appendix~\ref{ch:examples} to see two detailed examples. You can also find some other examples  a
 
 \url{https://github.com/corneliuhoffman/spatchcoqocaml/tree/master/examples}
\paragraph{\bf Warning for teachers}\label{subsec:warnings}

Spatchcoq is developed on top of Coq and  will suffer from some of the difficulties inherent in using Coq (or any other proof assistant). Proof assistants are based on type theory rather than set theory (see Appendix E for details) and this creates many of the issues. In particular every object in Coq has exactly one type and you need a conversion to move form one to the other. We have tried to keep this under the hood as much as possible and to mimic the usual set theory notations.

This creates certain surprising complications. For example, in the number theory chapter, we have decided to work with natural numbers rather than integers (this is mainly so that induction will work as expected). In general this is ok but, because N is not a ring, subtraction is a delicate beast. In particular, if m ? n are two natural numbers then m ? n = 0. Soa statement like (m?n)+n = m is only true if m ? n. For example (2?3)+3 = 0+3 = 3. For this reason we advise to tilt your examples in the number theory section towards addition rather than
subtraction. 

We will try to remind the reader about such issues as we go along. We will use the following format:\warn{Watch for the complications related to minus.}

We will also use some separate coloured boxes to separate txt. More neutral remarks are going to look like this:
\rmk{This is a remark that should help you understand the paragraph better.}


The text that is related to Spatchcoq will also be separated from usual text. The text that we input in to Spatchcoq will appear in input boxes like this:
\inp{
Lemma ancomm$(P\ Q:Prop):P\lor Q ->Q\lor P.$\\
Assume $(P \lor Q)$ then prove $(Q \lor P)$.
Consider cases based on disjunction in hypothesis Hyp.}

The goals that Spatchcoq will return to us will look like this:
\coq{Hyp:not\ (P \land (not\ Q))\\
Hyp0:(((not\ P) \land Q) \lor (P \land (not\ Q)))\\
}{Q}

And the messages that Coq will offer look like this:

\mess{error}


 

